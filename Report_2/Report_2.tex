\documentclass{article}
\usepackage[backend=biber, sorting=none]{biblatex}
\usepackage{graphicx}
\addbibresource{uni.bib}

\author{Angelo Battaglia}
\title{Internship Argo Software, report 2}
\begin{document}

\maketitle

\begin{center}
\includegraphics[width=3in]{./images/polito.jpg}
\includegraphics[width=3in]{./images/og_logoargosoft.png}
\end{center}

\section{The Time Table Problem in the Italian School System}
By citing \textcite{colorni1992genetic} let's start formalizing 
the model of the Italian School System.


\subsection{Teachers}
Teachers have an eighteen hours workweek. These eighteen hours are
considered to be stored into a number of didactic units, one of which
might be considered to be composed of one or more hours. In the simulation
these units are being called U.D. (unità didattiche). The variable totUD is
the total of these didactic units, while minUD and maxUD represent respectively
the minimum and the maximum, of that teacher for that specific class. Therefore, 
teachers may teach one or more subjects, in two or more classes. 

\subsection{Classes}

\section{Genetic Algorithm Introduction}
Genetic algorithms are a class of optimization algorithms inspired by the 
dynamics of evolution. The approach follows the survival of the fittest heuristics.
Given a number of individual outcomes, called generation, each one of these inherits 
valuable traits, useful for survival, from its ancestor. Each specimen belonging to a
single population may vary, having traits that will slow the survival will eventually
lead to the interruption of that speciment.  


\printbibliography[title = {Bibliography}]

\end{document}
